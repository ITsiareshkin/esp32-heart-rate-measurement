\documentclass[a4paper, 12pt,h]{article}
\usepackage[IL2]{fontenc}
\usepackage{times}
\usepackage{graphicx}
\usepackage{url}
\usepackage[a4paper, total={17cm, 24cm}, top=3cm, left=2cm]{geometry}
\usepackage[czech]{babel}
\usepackage{floatrow}
\setcounter{tocdepth}{5}
\usepackage{hyperref}
\usepackage[table,xcdraw]{xcolor}
\linespread{1.2}
\begin{document}

\begin{titlepage}
    \begin{center}
        \textsc{\Huge{Vysoké učené technické v Brně\\[0.3em]}}
        \textsc{\huge{Fakulta informačních technologií}}    
        \vspace{\stretch{0.382}}      
        
        \textsc{\Huge{Mikroprocesorové a vestavěné systémy\\[0.3em]}}
        
        \textbf{\textsc{\LARGE{ESP32: Měření srdečního tepu [digitální senzor]}}}        
        \vspace{\stretch{0.618}}
    \end{center}
    {   
        \Large \textbf{Ivan Tsiareshkin, xtsiar00} \quad \quad \quad \quad \quad \quad \quad \quad \quad \today 
    }
\end{titlepage}

\tableofcontents

\newpage

\section{Úvod}
Cílem projektu byl návrh a implementace vestavné aplikace, která měří srdeční tep a výsledek zobrazí na displej. Byla využitá deska WeMos D1 R32\footnote{\href{https://www.fit.vutbr.cz/~simekv/IMP\_projekt\_board\_ESP32\_Wemos\_D1\_R32.pdf}{https://www.fit.vutbr.cz/simekv/IMP\_projekt\_board\_ESP32\_Wemos\_D1\_R32.pdf}}, která obsahuje čip ESP32, modul pro měření srdečního tepu MAX30102\footnote{\href{https://www.hwkitchen.cz/max30102-snimac-pro-srdecni-tep-a-pulzni-oxymetr/}{https://www.hwkitchen.cz/max30102-snimac-pro-srdecni-tep-a-pulzni-oxymetr/}} a grafický OLED displej\footnote{\href{https://www.hadex.cz/m508a-displej-oled-096-128x64-znaku-7pinu-bily/}{https://www.hadex.cz/m508a-displej-oled-096-128x64-znaku-7pinu-bily/}}.

\subsection{OLED displej}
Grafický displej OLED 0,96" má rozlišení displeje 128x64 bodů a 7 pinů. Tato verze displeje komunikuje na rozhraní SPI. Pracovní napětí je 3.3 nebo 5V.

\subsection{Senzor MAX30102}
Senzor MAX30102 dokáže měřit srdeční tep i okysličení krve. Měření probíhá neinvazivně optickou metodou. Komunikace probíhá na rozhraní I2C. Podrobné informace o tom, jak senzor funguje, jsou \href{https://www.hwkitchen.cz/max30102-snimac-pro-srdecni-tep-a-pulzni-oxymetr/}{zde}.

\subsection{Zapojení}
Nejprve bylo VCC(3.3V) a GND z desky Wemos napojeno na "breadboard bus strips" (+ a -). Piny SCL a SDA byly použity pro I2C komunikaci mezi senzorem MAX30102 a deskou Wemos D1 R32. OLED displej komunikuje s deskou na rozhraní SPI a proto byly použity odpovídající piny. Konkrétní zapojení je vidět na \href{https://drive.google.com/drive/u/1/folders/1VMrUQWalsoRiYowo8jg8U5sH7tqxdGtw}{videu}.

\begin{table}[ht]
    \begin{tabular}{p{2.7cm} p{2.3cm} p{2.3cm}}
        \hline
        Periferní pin & ESP32 pin & Název v kódu\\ \hline
        Senzor VCC     & 3V3  & -- \\
        Senzor GND     & GND  & -- \\
        Senzor SCL     & SCL(IO22) & -- \\
        Senzor SDA     & SDA(IO21) & -- \\
        OLED VCC       & 3V3  & -- \\
        OLED GND       & GND  & -- \\
        OLED D0        & IO18 & OLED\_CLK  \\
        OLED D1        & IO23 & OLED\_MOSI \\ 
        OLED RES       & IO17 & OLED\_RESET\\ 
        OLED DC        & IO16 & OLED\_DC   \\ 
        OLED CS        & IO5 & OLED\_CS    \\ 
    \end{tabular}
    \caption{Hardwarové zapojení}
\end{table}

\newpage

\section{Řešení}
Řešení je implementováno v jazyce C v prostředí Arduino IDE s využitím knihoven pro MAX30102\footnote{\href{https://github.com/sparkfun/SparkFun\_MAX3010x\_Sensor\_Library}{https://github.com/sparkfun/SparkFun\_MAX3010x\_Sensor\_Library}} a OLED displej\footnote{\href{https://github.com/lexus2k/ssd1306}{https://github.com/lexus2k/ssd1306}}\footnote{\href{https://github.com/adafruit/Adafruit-GFX-Library}{https://github.com/adafruit/Adafruit-GFX-Library}}.

\subsection{Instalace a spuštění}
Projekt vyžaduje ke spuštění Arduino IDE\footnote{\href{
https://www.arduino.cc/en/software}{
https://www.arduino.cc/en/software}} verze 1.8.19 nebo vyšší. Pak je nutné přidat desku esp32 prostřednictvím Board Manageru a nastavit položku Board na ESP32 Dev Module. Dále je potřeba nainstalovat knihovny v Tools-Manage Libraries :
\begin{enumerate}
    \item SparkFun MAX3010x Pulse and Proximity Sensor Library: pro senzor měření tepu.
    \item Adafruit GFX Library (a potřebné komponenty): graphics core knihovna pro OLED displej
    \item Adafruit SSD1306: driver knihovna pro OLED displej
\end{enumerate}
Po nastavení Upload Speed na hodnotu 115200 a Portu projekt je možné spustit. Pokud se po kompilaci a nahrání projektu na desku rozsvítí u senzoru červená LED a na OLED displeji je napsáno přiložit prst na senzor, program běží a nainstaloval se správně.

\subsection{Programová část aplikace}
\begin{itemize}
    \item Na začátku kódu jsou připojeny potřebné knihovny, jsou nastaveny velikosti displeje a hodnoty jeho pinů. Rovněž jsou vytvořeny 2 bitmapové obrázky srdce, z nichž jeden je větší, při čtení dat budou tyto obrázky střídavě zobrazovat na obrazovce, napodobující srdeční tep. Pro výpočet hodnot srdečného tepu jsou také vytvořeny globální proměnné.
    \item Funkce setup() inicializuje senzor a displej a v případě selhání vypíše chybu. Displej se vymaže a senzor se rozsvítí červeným LEDem, což znamená, že senzor běží.
    \item Funkce loop() obsahuje programový kód, který bude opakovaně prováděn v nekonečné smyčce. Konkrétně následující: senzor načte hodnotu IR a na základě ní program rozhodne, zda je prst na senzor přiložen. Pokud ano, na displeji se zobrazí malý symbol srdce a průměrná hodnota srdečního tepu. Dále po detekci jednoho srdečního tepu program vypočítá průměrnou hodnotu srdečního tepu za minutu(BPM) z rychlosti posledních 4 tepů a zobrazí tuto hodnotu na OLED displeji spolu s velkým symbolem srdce. Pokud je hodnota IR malá, zobrazí se na OLED displeji žádost o přiložení prstu na senzor.
\end{itemize}

\newpage
\section{Shrnutí}
Při psaní tohoto projektu byly implementovány všechny požadavky uvedené v zadání. Autoevaluace složek E, F, Q, P, D je 14. \\ 
Při testování implementace byly výsledky měření porovnány s výsledky měření chytrých hodinek Apple Watch a výsledky lišily jenom nepatrně.

\section{Příloha}
Video: \href{https://drive.google.com/drive/u/1/folders/1VMrUQWalsoRiYowo8jg8U5sH7tqxdGtw}{https://drive.google.com/drive/u/1/folders/1VMrUQWalsoRiYowo8jg8U5sH7tqxdGtw}

\end{document}
